%!TEX root = Main.tex
\documentclass[Main]{subfiles}

\begin{document}
\section{Theory and Related Works} % (fold)
	\label{sec:theory_and_related_works}
	Humans are very good at discerning emotions in others from looking at pictures or sequences of pictures of faces, among these, the sensation of pain.
	This tells us that there must be visual cues in these, that it is possible to detect and use for classification.



	\subsection{The UNBC-McMaster Shoulder Pain Expression Archive Database} % (fold)
		\label{sub:unbc_mcmaster}
		
		\subsubsection{Pain Score} % (fold)
			\label{ssub:pain_score}
			
			% subsubsection pain_score (end)

		% subsection unbc_mcmaster_schoulder_pain_database (end)

	\subsection{Related Works} % (fold)
		\label{sub:related_works}	

		\fxnote{More articles }

		\fxnote{Insert authors name} \cite{Ashraf2009} use the UNBC database.
		They use the registered landmarks in the images to make procrustes aligned representations of the faces along with facial appearance, shape normalized using Active Shape Models (ASM) and Active Appearance Models (AAM), as feature vectors.
		These feature vectors are then used to train a Support Vector Machine (SVM).
		With this they achieved detection rates of $57.1\%$ to $87.5\%$ for individual FACS AUs and up to $83.9\%$ for binary pain detection.

		In this report I attempt to reproduce these results using the same data and with a similar approach but with my own implementation.

		\fxnote{Insert names here} \cite{Kaltwang2012} similarly use the UNBC database.
		They compare the performance of using shape based features (PTS), taken from the registered landmarks, with appearance features.
		The appearance feature were both 2D Discrete Cosine Transform (DCT) and Local Binary Patterns (LBP).
		The authors employ Relevance Vector Regression (RVR) to estimate pain for type of feature (shape, DCT and LBP).
		A fusion of the individual features pain estimates is then made, again using RVR.
		The authors conclude that appearance based features outperform shape based.
		They do however believe that improved registration of shape, that compensates for out of plane motion (eg. \cite{Rudovic2011}), would improve performance.
		Regarding the appearance based feature, the authors state that in most cases the LBP features worked better than DCT features, taken individually.





		A major shortfall of the methods described above is that they all \fxnote{Check if this is true when other articles are added} rely on the facial landmarks that are registered with human intervention.
		Without self implementing a strong and reliable way of automatically registering these, systems like the ones described above are not applicable for use in real-life, real-time applications.
		Therefore I try a completely different approach in this report. 

		% subsection related_works (end)

	\subsection{Proposed approach} % (fold)
		\label{sub:proposed_approach}
		The novel approach in this project is to use only the image data from the UNBC database as input data and only use the metadata for ground truth labeling of the training, test and validation data.

		The method can be summarized as follows:
		\begin{enumerate}
			\item 
			A Viola-Jones detector \fxnote{Reference?} will be used to locate the face in the images.
			
			\item 
			A patch around the face will be extracted.

			\item
			The extracted patch will be resized to a fixed size eg. $100 \times 100$ or $48 \times 48$ pixels.

			\item
			The resized patches will be used to train a Convolutional Neural Network (CNN)

		\end{enumerate}

		Several ideas for the scope of each classification network exist.
		One idea is to train a large network with all the data on all types of subject, to have a very generalized model.

		Another is to train a couple of networks one a few stereotypes of subjects ie. males, females, children and perhaps also different age segments separately. This would be less general, but maybe better tuned to the idiosyncrasies of each group.

		A final idea is to train a model on a per subject basis.
		This would likely not generalize very well to all other subjects.
		Instead it may be better equipped to handle the sometimes very individual response of subjects to certain levels of pain.
		\fxnote{Make sure to include something about handicapped/paralyzed subjects} 
		This idea may be especially relevant in cases with subjects who are mentally handicapped or paralyzed, as their response patterns related to pain are know to deviate from the norm.\fxnote{Reference}
		For this to work though, it would require that it is possible to train the network with few enough data, that it would be feasible to acquire for each subject.
		A way of achieving this would be to refine a more general pre-trained network.


		% subsection proposed_approach (end)
	
	% section theory_and_related_works (end)

\end{document}
