%!TEX root = Main.tex
\documentclass[Main]{subfiles}

\begin{document}
\section{Introduction} % (fold)
	\label{sec:introduction}

	\subsection{Purpose and Goal} % (fold)
		\label{sub:purpose_and_goal}
		The purpose of this R\&D project is to examine the possibility of creating a system that automatically and objectively  assesses pain in humans by observing their faces in pictures or short video sequences.

		There are several goals for this project.
		These are:
		\begin{enumerate}
			\item 
				To examine the literature on the state of the art of pain assessment with vision systems in order to assess which methods and features seem promising.
			\item
				Gain insight into the theory of machine learning with deep artificial neural networks or \emph{Deep Learning}.
			\item
				Attempt to implement a system applying deep learning to the problem of automatic visual pain assessment.
			\item 
				Comparing the above mentioned system to the state of the art using a publicly available database of images for pain assessment.
		\end{enumerate}

		% subsection purpose_and_goal (end)

	\subsection{Scope} % (fold)
		\label{sub:scope}
		This project will be carried out as an \emph{Engineering Research and Development Project (TIENPRAU01)} and as such has a scope of \emph{5 ECTS} points over \emph{one school quarter}.

		% This means that the scope of the study into the state of the art of systems for pain assessment with computer vision will be limited to reviewing just a handful of essential papers.
		This means that the scope of the study into the state of the art of systems for pain assessment with computer vision will be limited to reviewing just a few essential papers.

		Further, a system for pain assessment with computer vision using deep learning will only be attempted implemented in MATLAB as a proof of concept.

		% subsection scope (end)

	% section introduction (end)

\end{document}