% !TEX root = Main.tex
\documentclass[Main]{subfiles}

\begin{document}
\section{Normalized Appearance and Shape Method Using SVM} % (fold)
	\label{sec:normalized_appearance_and_shape_method_using_svm}
	In this section I describe how I tried to reproduce the results of \cite{Lucey2011} by implementing my own algorithm for extracting data and training a linear Support Vector Machine (SVM) using LIBSVM \cite{CC01a}.

	The data extracted from the database and used in the experiments are normalized shape and shape normalized appearance (described in greater detail in Section \ref{sub:data_pre_processing_and_feature_extraction} below).
	Two experiments will be done with these data.
	One will focus on detecting the presence of FACS Action Units relevant to PSPI model described in Section \ref{ssub:pain_score}.
	The other will focus only on the detection of pain.
	Both will experiments will tested using the shape data alone and with shape and appearance data combined.

	% Method

		% Procustes alignment of faces
		% Delauney triangulation of mean shape
		% New triangulation of aligned faces
		% Affine warp to mean shape
		% Extract and vectorize appearance
		% 

		
	% My implementation
	% My Results
	% Discuss, compare and reference results
		% Strong points
		% Shortcomings

	\subsection{Data pre-processing and Feature Extraction} % (fold)
		\label{sub:data_pre_processing_and_feature_extraction}

		\subsubsection{Shape Data} % (fold)
			\label{ssub:shape_data}
			The raw shape data consists of \texttt{.txt}-file for each frame with two columns of (x,y) pixel coordinates, 66 points in total each for a specific facial landmark.
			\fxnote{insert image of facial landmarks, possibly numbered} 
			These combined make a mask of the facial shape.
			They are however centered in the corner of the image and shifted, scaled and rotated arbitrarily form subject to subject and from frame to frame because of movement.

			\paragraph{Procrustes Alignment} % (fold)
				\label{par:procrustes_alignment}
				In order to use the shape data they need to be shifted to a zero mean reference and be normalized with regard to size and rotation, without affecting shape.
				For this, Procrustes alignment is just what is needed.
				Procrustes alignments works by aligning sets of points to a common reference (either given, or taken as the mean of the set to be aligned) by a rigid transformation ie. translation, rotation and scale.

				In my project I do this iteratively and by using MATLABs build in \texttt{procrustes} function, like so:
				\begin{enumerate}

					\item
					For each shape $S_i$ subtract the mean of all points in $S_i$.

					\item
					Calculate mean shape $Z_0$, by taking the mean over all shapes $S_i$ for each point $S_{i,j}$.

					\item
					\label{enum:goto}
					Align all shapes $S_i$ to $Z_0$ using \texttt{procrustes} to get the aligned shapes $Z_i$.

					\item
					Calculate the new mean shape $Z_0$ over all shapes $Z_i$ for each point $Z_{i,j}$.

					\item
					Repeat from step \ref{enum:goto} for as many times as is needed. 
					A typical stopping criteria is when $Z_0$ stops changing significantly.

				\end{enumerate}
				Two iterations will typically suffice and is therefore what is used in this project.

				% paragraph procrustes_alignment (end)

			% subsubsection shape_data (end)


		\subsubsection{Appearance Data} % (fold)
			\label{ssub:appearance_data}
			
			\fxnote{some text}
			% Need for common shape
			% How to preserve appearance
			% Something with eyes
			% OWN IMPLENTATION

			\paragraph{Delauney Triangulation} % (fold)
				\label{par:delauney_triangulation}
				
				% paragraph delauney_triangulation (end)

			\paragraph{Piecewise Affine Warp} % (fold)
				\label{par:piecewise_affine_warp}
				
				% paragraph piecewise_affine_warp (end)







			% subsubsection appearance_data (end)

		\subsubsection{Combining Data} % (fold)
			\label{ssub:combining_data}

			% Vectorization
			
			% subsubsection combining_data (end)
		
		% subsection data_pre_processing_and_feature_extraction (end)


	\subsection{Experiment 1: Recognizing FACS Action Units} % (fold)
		\label{sub:experiment_1_recognizing_facs_action_units}
		
		% subsection experiment_1_recognizing_facs_action_units (end)


	\subsection{Experiment 2: Detecting Pain in faces} % (fold)
		\label{sub:experiment_2_detecting_pain_in_faces}
		
		% subsection experiment_2_detecting_pain_in_faces (end)


	\subsection{Discussion} % (fold)
		\label{sub:discussion}
		
		% subsection discussion (end)

	% section normalized_appearance_and_shape_method_using_svm (end)


\end{document}