%!TEX root = Main.tex
\documentclass[Main]{subfiles}

\begin{document}
\section{Implementation Code} % (fold)
	\label{sec:code}
	The implemented code for this project and some data will be available in digital form upon request to the author.



	% section code (end)

\newpage
\section{GPU Acceleration of CNN training} % (fold)
	\label{sec:gpu_acceleration_of_cnn_training}
	The CNN toolbox MatConvNet from \texttt{VLfeat.org} is MATLAB toolbox that is implemented efficiently in C++ and in CUDA for GPU acceleration.
	In order to use it, it is necessary to compile the code locally.
	Fortunately the toolbox contains MATLAB scripts that handle all compiling and linking of files for you.

	\subsection{Compiling for CPU} % (fold)
		\label{sub:compiling_for_cpu}
		There are some prerequisites to be able to compile the toolbox:
		\begin{itemize}
			\item
			MATLAB version 2013b or higher.

			\item
			C++ compiler
			\begin{itemize}
				\item
				Microsoft Visual Studio 2010 or newer

				\item
				Alternatively one can use the Microsoft Visual C++ Redistributable. Both x86 and x64 (32-bit and 64-bit respectively) versions is needed.
			\end{itemize}

		\end{itemize}

		This will allow you to compile the toolbox for CPU operation. Simple navigate to the directory where your toolbox is unpacked, add the \emph{matlab} folder to the MATLAB path and run the script \texttt{vl\_compilenn} with no arguments

		> \texttt{cd <MatConvNet>}\\
		> \texttt{addpath matlab}\\
		> \texttt{vl\_compilenn}\\

		If compilation completes with no errors, call \texttt{vl\_test\_nnlayers} with no arguments to test the compiled code.

		> \texttt{vl\_test\_nnlayers}\\

		% subsection compiling_for_cpu (end)

	\subsection{Compiling for GPU} % (fold)
		\label{sub:compiling_for_gpu}
		In order to compile for GPU acceleration there are some further prerequisites.
		\begin{itemize}
			\item
			An NVIDIA GPU with \emph{Compute Capability} of 2.0 or higher. The 600 series and higher should do.

			\item
			A version of the NVIDIA CUDA Toolkit (Available at \cite{Corporation2015}).
			The authors of MatConvNet have some recommendation for which version depending on MATLAB version at \cite{Lenc2014}.
		\end{itemize}

		\newpage
		To compile for GPU, \texttt{vl\_compilenn} is again called but this time with some arguments.

		> \texttt{vl\_compilenn('enableGpu', true, 'cudaRoot', '/Developer/NVIDIA/CUDA-6.0')}

		This uses MATLAB build-in \texttt{mex} compiler to compile the code. The \texttt{cudaRoot} argument is optional, and tells the script where to find your CUDA toolkit.
		You may experience problems with this depending on your combination of MATLAB version and CUDA Toolkit.
		In this case you need to compile using \texttt{nvcc} (NVIDIA CUDA Complier).
		To do this call \texttt{vl\_compilenn} like this instead:

		> \texttt{vl\_compilenn('enableGpu', true, ...\\
			\-\hspace{3.05cm}'cudaRoot', '/Developer/NVIDIA/CUDA-6.5', ...\\
			\-\hspace{3.05cm}'cudaMethod', 'nvcc')}

		Here I did find a problem with the (as of this writing) current version of the toolbox.
		In the script \texttt{vl\_compilenn} the arguments pertaining to the GPU compute architecture is not send to the compiler when using \texttt{nvcc}.
		This causes the compiler to compile with full backwards compatibility in mind, causing the process to fail, as some necessary functions are available on old architectures.
		I solved this by adding a line to the script (See line 305 in Listing \ref{lst:vl_compilenn_exerpt})

		\begin{lstlisting}[caption=Exerpt from \texttt{vl\_compilenn}, style=Code-Matlab, label=lst:vl_compilenn_exerpt]
300	% For the cudaMethod='nvcc'
301	if opts.enableGpu && strcmp(opts.cudaMethod,'nvcc')
302		flags.nvcc = flags.cc ;
303		flags.nvcc{end+1} = ['-I"' fullfile( ...
											matlabroot, ...
											'extern', ...
											'include') '"'] ;
304		flags.nvcc{end+1} = ['-I"' fullfile(...
											matlabroot, ...
											'toolbox', ...
											'distcomp', ...
											'gpu', ...
											'extern', ...
											'include') '"'] ;
305		flags.nvcc{end+1} = opts.cudaArch ; % Correction // KN 2015-03-30
306		if opts.debug
307			flags.nvcc{end+1} = '-O0' ;
308		end
309		flags.nvcc{end+1} = '-Xcompiler' ;
310		switch arch
311			case {'maci64', 'glnxa64'}
312				flags.nvcc{end+1} = '-fPIC' ;
313			case 'win64'
314				flags.nvcc{end+1} = '/MD' ;
315				check_clpath(); % check whether cl.exe in path
316		end
317	end
			\end{lstlisting}


		If you have already compiled for CPU, you may experience some conflicts.
		If so, you need to delete the contents of the folder \texttt{/matlab/mex} in the toolbox directory before compiling for GPU.

		% subsection compiling_for_gpu (end)

	% section gpu_acceleration_of_cnn_training (end)

\end{document}